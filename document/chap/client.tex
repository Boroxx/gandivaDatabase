\chapter{Client}
\label{Client}


\section{Aufbau des Clients}

Der Client wird verwendet um mit der Datenbankanwendung per REST-API kommunizieren zu können. Er basiert auf einer Shell-Anwendung. Somit kann gezielt und schnell eine Datenanfrage über die Kommandozeile losgelöst werden.
Im folgenden wird kurz auf die Architektur des Clients eingegangen und anschließend ein Anwendungsfall als Beispiel erklärt.

\subsection{Architektur}

Der Client basiert auf dem Spring Shell Framework, welches als Grundlage für eine Kommandozeilen-Anwendung dient. \cite{Spring:SpringShell}
Hier können ganz einfach verschiedene Kommandozeilen-Befehle auf interne Methode in der Anwendung gemapped werden.\\

Da die verschiedene Methoden auf die REST-API zugreifen müssen, wird hier mithilfe einer \texttt{HTTPClient}-Klasse verschiedene GET- und POST-Requests ausgelöst, welche die Datenbankanwendung entgegen nimmt.


\subsection{Anwendungsfall}

Vorerst muss auf Seiten des Servers die Datenbankanwendung eingerichtet werden. Dazu liegt eine bootable Jar-Datei im Gitlab-Repository. Diese sollte im gleichen Ordner, wie die verschiedenen SQL-Dumps ausgeführt werden.\\
Um die Jar-Datei auszuführen reicht folgender Befehl:

\begin{center}
\code{java -jar datenbankanwendung.jar}
\end{center}

Im gleichen Ordner liegt ebenfalls eine \texttt{client.jar}, welche als Argumente einen Host, sowie einen Port, welcher auf die Datenbankanwendung zurückzuführen ist, übergeben bekommt.

\begin{center}
\code{java -jar client.jar --db.port=8080 --db.host=localhost}
\end{center}

Sobald die Datenbankanwendung und der Client ausgeführt wurden, kann man im Client den Befehlt \textbf{help} ausführen. Unter der Überschrift \texttt{Database Commands} findet man alle nötigen Befehle die man für die Initialisierung, sowie das absenden von Anfragen, benötigt.

Um beispielsweise ein Statement abzusenden, kann man folgenden Befehl verwenden. 

\begin{center}
\code{send \grqq{}SELECT * FROM table\grqq{}}
\end{center}

Es muss jedoch darauf geachtet werden, dass die Datenbankanwendung erstmalig mit dem richtigen SQL-Dump initialisiert wurde. Als Antwort auf eine abgeschicktes Statement, bekommt man hier die \texttt{IndexResponse} in Form eines JSON-Strings zu sehen. Dies Dient zur Demonstration für den späteren Gebrauch von RDMA.



