\chapter{Gliederung der Arbeit}
\label{gliederung}

Eine sinnvolle und korrekte Unterteilung der Arbeit ist nicht nur wichtig für den Leser, sondern hilft auch dem Verfasser bei der Anfertigung.\\
Allgemeiner grundlegender Aufbau mit den wichtigsten inhaltlichen Aspekten:

\begin{enumerate}
 \item Einleitung: Hinführung an die Thematik, Problemstellung, Ziel der Arbeit, Aufbau der Arbeit
 \item Grundlagen: Erläuterung von Begriffen, Definitionen, Algorithmen, Programmiertechniken, etc. die im Kontext der 
 Arbeit Verwendung finden und welche der Leser für das Verständnis benötigt. Hierbei muss nicht ein ganzer Themenbereich der Informatik
 ausgearbeitet und erklärt werden. Referenzen auf Literatur sind hier sehr wichtig.
 \item Hauptteil: Der Hauptteil der Arbeit besitzt keine vorgegebene Gliederung. Hier muss eine dem Thema angemessene Unterteilung in 
 ein oder mehrere Kapitel und Unterkapitel ausgearbeitet werden. Oft bietet sich hier eine Trennung von Design/Architektur und Implementierung an.
 \item Evaluation: Messungen von selbstgeschriebenen oder bereits vorhanden Benchmarks müssen in diesem Abschnitt mithilfe von Tabellen und Grafiken dokumentiert
 werden. Weiterhin muss der Versuchsaufbau festgehalten werden. Dabei sind alle in den Messungen verwendeten Systeme mit ihren (relevanten) Hardware- und Softwarekonfiguration
 aufzuführen. Die Ergebnisse müssen bewertet und ausführlich erklärt werden. Ausreisser oder Anomalien sind nicht außer Acht zu lassen!
 \item Zusammenfassung, Fazit und Ausblick: Zum Schluss werden die vorgestellten Konzepte und die Implementierung zusammengefasst. 
 Ein Bezug auf die erfüllten/nicht erfüllten Ziele der Arbeit sowie die Ergebnisse der Evaluation dürfen hier nicht fehlen. 
 Weiterführende Ideen und Konzepte, die über die Themenstellung hinausgehen und Verbesserungspotential sind mögliche Punkte für einen Ausblick.
\end{enumerate}
