\chapter{Grundlagen}

\section{Datenbankverwaltungssysteme}

Einer der wichtigsten Aspekte einer Datenbankanwendung ist die Performance. 

Mit der Zeit wird es auf der Welt immer wichtiger große Mengen an Daten mithilfe einer Datenbank persistieren und verarbeiten zu können.
In Bezug auf den Speicher kann hier unterschieden werden zwischen einer In-Memory Datenbank,wie beispielsweise Redis oder H2 und einer On-Disk Datenbank,wie Postgresql,welche auf dem Sekundärspeicher operiert. \\

In-Memory Datenbanken haben den Vorteil, dass diese eine geringe Latenz bei Anfragen und Operationen vorweisen, da die Daten sich schon im Hauptspeicher befinden. Bei einer On-Disk Datenbank muss der benötigte Speicher bei einer Anfrage erst in den Hauptspeicher aus dem Sekundärspeicher geladen werden. Diese Input/Output Operationen konsumieren eine Menge Zeit bei einer großen Menge von Daten.\cite{KABAKUS2017520}
Aufgrund dieser Tatsache ist es für diese Thesis sinnvoll sich In-Memory Datenbankverwaltungssystem genauer anzuschauen. 
Vorerst muss jedoch geklärt werden, welche Arten es von Hauptspeichern gibt und welche Speichertechnologien die Zukunft mit sich bringt.

\section{Speichertechnologie}

\subsection{Hauptspeicherarten}

Es gibt mehrere Arten von Hauptspeicher. Eine der gängigsten Arten sind Dynamic Random Memory Access abgekürzt DRAM und Static Random Memory Access abgekürzt SRAM.Im Gegensatz zu DRAM ist SRAM deutlich teurer,da es um ein Datenbit zu halten mehrere Transistoren benötigt.SRAM steht gegenüber DRAM jedoch Performance-technisch in Bezug auf die Geschwindigkeit besser dar.\cite{techtarget:Ram}

\subsection{Neue Speichertechnologie}
Durch die immer größer und immer schneller werdende Entwicklung von digitalen Medien und Dienstleistungen, steigt auch die Größe an produzierten Daten. 
Das Unternehmen Intel publizierte 2015 einen Artikel "Intel and Micron Produce Breathrough Memory Technology" in dem sie die Erfindung der "3D Xpoint" Technology bekannt gaben. Diese Technologie brachte eine neue Speicherkategorie auf den Markt. Es handelt sich hierbei um nicht flüchtigen Speicher welcher 1000 mal schneller sein soll, als der gängige Sekundärspeicher. \cite{Intel:MemoryTechnology}

Mit diesem Artikel kam die Hardware Intel Optane Persistent Memory auf den Markt, welche eine Kapazität von bis zu 512 GB an Speicher aufweisen kann.
Im Gegensatz dazu befindet sich DRAM bei ca 128GB.
Diese neuen Speichermodule sind DDR4 kompatibel und somit als Hauptspeicher nutzbar. \cite{IntelOptane:Micron}

Gegenüber DRAM hat das Intel-Optane-Modul Nicht-flüchtigen Speicher. Das bedeutet,dass  Daten welche sich in einer Datenbankverwaltung befinden,bei Spannungsverlust nicht verloren gehen.
Die Eigenschaften des IntelOptane Moduls sollen es ermöglichen eine Große Menge an Daten im Speicher abzulegen und trotz der Verwendung von Nicht-flüchtigem Speicher eine gute Performance sicherzustellen.
 
\subsection{Reihen-orientierter Speicher vs Spalten-orientierter Speicher}
Bei der Performance spielt auch eine große Rolle,in welchem Format die Daten in den Speicher gelegt werden. Hier wird unterschieden zwischen dem Reihen-orientierten und dem Spalten-orientierten Speicherformat.
Die meisten traditionellen Datenbankverwaltungssysteme benutzen ein Reihen-orientiertes Speicherformat, welches Datentupel aus den Reihen hintereinander im Speicher persistiert. Solch ein System ist laut \cite{Stonebraker2005CStoreAC} ein "Schreib-optimiertes System".

Grafik

Bei einem Spalte-orientierten Datenbanksystem, werden keine Dateneinträge hintereinander im Speicher persistiert, sondern die Einträgen spaltenweise pro Eintrag nacheinander in den Speicher gelegt. In der Grafik sieht man, dass jede ID eines Eintrags hintereinander im Speicher liegt. Auf diese Weise können auf größeren Datensätzen schnellere Funktionen auf einzelne Einträge ausgeführt werden, ohne dass Daten, die nicht benötigt werden,mitgeladen werden müssen.

Im Gegensatz dazu gleicht der Ansatz eines Spalten-Orientierten Datenbankspeicher-Layouts diesen Performanceverlust aus. 
Da die Daten eines Objektes hier Spaltenweise persistiert werden können Operationen auf langen Spalten sehr performant ausgeführt werden.

Grafik

Aufgrund dieser Vorteile wird in dieser Arbeit ausschließlich eine In-Memory Datenbank mit einem Spalten-Orientierten Speicher behandelt.
Zur Speicherung der Daten in einem Spalten-orientierten Format wird Apache Arrow verwendet.

\section{Apache Arrow}

Apache Arrow ist ein Framework, welches ein Sprachen-unabhängiges Spalten-orientiertes Speicherformat definiert. 
Um Arrow nutzen zu können stellt Apache, in verschiedene Sprachen Bindings zu Verfügung. In dieser Arbeit wird die Programmiersprache Java benutzt.
Arrow umgeht das Problem, dass jede Sprache eine unterschiedliche Serialisierung und Deserialisierung von Daten im Speicher und in der Übertragung von Daten über das Netzwerk,unterstützt.
Das Framework vereinheitlicht die Serialisierung und Deserialisierung indem es sämtliche Objekte unter der Haube als Buffer interpretiert.
Auf ArrowBuffer wird im Hauptteil eingegangen.


\section{Serialisierung und Deserialisierung}

Was ist das?

\section{Gandiva}

\section{JSQL}


 



(https://db.in.tum.de/teaching/ws1718/seminarHauptspeicherdbs/paper/sterjo.pdf?lang=de)