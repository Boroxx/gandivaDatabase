\chapter{Einleitung}

\section{Problemstellung}
Ein Großteil von traditionellen Datenbankanwendungen basieren auf dem Reihen-orientierten Speicherformat. Für große Datensätze auf denen Operationen wie beispielsweise das Summieren von Spalten stattfinden soll, finden sich Performance-Einbussen.
Mit Hinblick auf die Entwicklung von nicht flüchtigem Speicher (Non Volatile Memory), werden Datenbanken die im Hauptspeicher agieren immer mehr von größerer Bedeutung. Daher ist es notwendig eine Datenbankverwaltung bzw. Anwendung zu implementieren, welche große Datensätze sinnvoll im Hauptspeicher persistiert und mit verschiedenen Anfragen abgefragt werden kann.
Vorausschauend soll diese Datenbankanwendung nach einer Anfrage von Daten, Indices in Form von virtuellen Speicheradressen zurückgeben. Damit mithilfe dieser Speicheradressen später mithilfe von RDMA auf die Datenzugreifen kann. Letzteres ist jedoch nicht Teil dieser Arbeit. 


\section{Ziel der Arbeit}
Ziel der Arbeit ist es eine Datenbankanwendung basierend auf Apache Arrow/Gandiva zu implementieren, welche unter der Haube mit einem Spalten-Orientierten Speicherformat funktioniert. Die Anwendung soll einen SQL-Dump in den Hauptspeicher einlesen können und auf Anfrage eines Client den Index der Virtuellen Speicher-Adresse zurückgeben. Auf der Datenbank sollen, mithilfe des Frameworks Gandiva, Exressions und Filter ausgewertet werden können.

Über eine Schnittstelle soll ein Client mit der Datenbankanwendung kommunizieren und Micro-Benchmarkings vornehmen können. Hier wird besonders Fokus auf die Zeitspanne der Initialisierung der Datenbank gelegt.


\section{Aufbau der Arbeit}

Die Arbeit besteht aus mehreren Kapiteln. Im zweiten Kapitel werden die Grundlagen besprochen, die das Basiswissen bereitstellen,um die darauffolgenden Kapitel wie z.B die Architektur und Implementierung, verstehen zu können.
In den Grundlagen werden besonders auf die Apache Arrow und Gandiva eingegangen,sowie auf das Grundwissen zu Speichertechnologien.\\
Im Kapitel \ref{Architektur} wird die Architektur der Datenbankanwendung und des Clients beschrieben. Darauf folgt das Kapitel \ref{Implementierung} , welches die Implmentierung der Architektur, sowie einige Features der Anwendung und das Zusammenspiel der Komponenten beschreibt. Außerdem wird in diesem Kapitel erklärt, wie man die Anwendung richtig aufsetzt und starten kann.
Anschließend werden verschiedene Messergebnisse im Kapitel \ref{Evaluierung} diskutiert und es wird ein Fazit über die Performance der Anwendung gezogen.
Abschließend gibt es eine Zusammenfassung ,sowie einen Ausblick auf die Datenbankanwendung und ihren Anwendungsfall, in Kapitel \ref{Zusammenfassung}. 
  